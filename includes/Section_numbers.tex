\section{Zahlenrepräsentation}
\label{sec:numbers}

Die Zahlenrepräsentation von \projectname{} folgt eng derer anderer Lispimplementierungen wie Common Lisp \cite[143 -- 144]{graham_ansi_1995}. Intern kennt \projectname{} drei verschiedene Zahlentypen: !long long!, !double! und !fraction!. Diese werden dynmisch unter Erhalt größtmöglicher Genauigkeit umgewandelt. Tabelle \ref{tab:typeconversion} zeigt den Ergebnistypen einer arithmetischen Operation in Abhängigkeit der Operandentypen.
\begin{table}[htbp]
	\centering
	\begin{threeparttable}
		\captionabove{Rückgabetyp arithmetischer Operatoren nach Operandentyp}
		\begin{tabular*}{0.6\linewidth}{l|lll}
					&long	&double	&fraction\\
			\hline
			long		&long	&double	&fraction\\
			double	&double	&double	&double\\
			fraction	&fraction&double	&fraction\tnote{*}\\
		\end{tabular*}
		\begin{tablenotes}
			\item{*}long falls das Ergebnis ganzzahlig ist
		\end{tablenotes}
		\label{tab:typeconversion}
	\end{threeparttable}
\end{table}
Intern werden Zahlen dazu durch die Klasse !number! verwaltet, die auf der Klasse AnyScalar von Timo Bingmann basiert \cite{anyscalar_online}. Alle Datentypen werden in einer !union! verwaltet, der aktuelle Typ wird durch einen Enumerator gespeichert. Die !number! Klasse übernimmt alle nötigen Typkonvertierungen sowie boolsche und arithmetische Operatoren durch zwei Templates. Der konkrete Operator wird dabei als Templateparameter übergeben. Listing \ref{lst:op_template} zeigt dies in verkürzter Form als Beispiel. Die !C++!-Standardbibliothek bietet dazu die Klassen !std::plus!, !std::minus!, !std::multiplies! und !std::divides!. Bei Fehlern wie der Division durch $0$ wird eine !arith_error!-Exception geworfen.
\begin{lstlisting}[caption={Arithmetische Operatoren}, label=lst:op_template]
template <template <typename Type> class Operator, char OpName>
number number::binary_arith_op(const number &b) const
{
    switch(atype)
    {
    case ATTRTYPE_LONG:
    {
        switch(b.atype)
        {
        case ATTRTYPE_LONG:
        {
            Operator<long long> op;
            if (OpName == '/') {
                if(b.val._long == 0)
                    throw arith_error("division by zero");
                if(val._long%b.val._long != 0) {
                    number b2(b);
                    b2.convert_type(ATTRTYPE_FRACTION);
                    number a2(*this);
                    a2.convert_type(ATTRTYPE_FRACTION);
                    Operator<fraction> op2;
                    return number(op2(a2.val._fraction, b2.val._fraction));
                }
            }
            return number( op(val._long, b.val._long) );
        }
        case ATTRTYPE_DOUBLE:
        {
            Operator<double> op;
            return number( op(static_cast<double>(val._long), b.val._double) )
        }

        case ATTRTYPE_FRACTION:
        {
            Operator<fraction> op;
            if (OpName == '/' && b.val._long == 0)
                throw arith_error("division by zero");
            number a(*this);
            a.convert_type(ATTRTYPE_FRACTION);
            return number(op(a.val._fraction, b.val._fraction));
        }
        }
        break;
    }

    //the same for atype == ATTRTYPE_DOUBLE and atype == ATTRTYPE_FRACTION

    }
    assert(0);
}

// Force instantiation of binary_op for the four arithmetic operators

/// Forced instantiation of binary_arith_op for number::operator+()

template number number::binary_arith_op<std::plus, '+'>(const number &b) const;
\end{lstlisting}

Die Klasse !number! ist wie alle Lispobjekte von !object! abgeleitet; zur einfacheren und schnelleren Typprüfung überschreibt sie die Methode !is_number()!, sodass sich Zahlen leicht durch Aufruf dieser Methode erkennen lassen. Beispiele zur Verwendung der !number! Klasse und Implementierung der arithmetischen Funktionen finden sich in Abschnitt \ref{sec:cxx_function_interface}.

Wie bereits in \ref{sec:shared_ptr} beschrieben werden sämtliche Objekte durch !shared_ptr! verwaltet. Um eine bequemere Verwendung auf diese Weise referenzierter !number!-Objekte zu ermöglichen existiert der !typedef! !number_ptr_t! --- eine Spezialisierung von !shared_ptr!.

Im Quelltext werden Zahlen als Ganzzahl, Fließkommazahl und Bruch erkannt. Der Tokenizer erzeugt !number!-Objekte und passt den Typ entsprechend an. Auf die genaue Funktionsweise geht Abschnitt \ref{sec:tokenizer} ein.