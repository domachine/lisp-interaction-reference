\section{Skriptverarbeitung}
%mir fällt beim besten willen kein gescheiter Titel ein
%vielleicht Ausführungsschritte?
%oder Laufzeit[irgendwas]?

\subsection{Der Tokenizer}
\label{sec:tokenizer}

%weiß nicht, ob diese Section notwendig ist, aber das Label wird referenziert…
%hab eigentlich keine Ahnung von der Materie, also nimm mir Ungenauigkeiten/Fehler nicht übel
%das ganze ist bisher auch eher Alibimäßig, muss also entweder erweitert oder gelöscht werden.
Der Tokenizer dient der lexikalischen Analyse und zerlegt den Eingabestrom in einzelne Tokens, also logisch zusammenhängende Einheiten. In \projectname{} sind das:
%Nicht nötig, aber perfekt um gelerntes zu präsentieren…
\begin{itemize}
\item{Klammern}
\item{Strings}
\item{Symbole}
\item{Konstanten}
\end{itemize}
%als Beispiel vielleicht der reguläre Ausdruck um Zahlen von Symbolen zu unterscheiden
Der Tokenizer überprüft dabei nur die syntaktische Korrektheit, die semantische Analyse beziehungsweise die Interpretation führt der nachgeschaltete Interpreter aus.

%hier vielleicht noch was zum Interpreter…
\subsection{Der Interpreter}
