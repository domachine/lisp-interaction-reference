\section{Environment}
\label{sec:environment}

Die !lisp::environment! Klasse ist eines Herzstücke von \projectname{}.
Sie stellt eine Umgebung zur Verfügung, die von lisp-typischen Formen wie !let! und !lambda!
eingeleitet wird. Sie hält eine sogenannte Hashtable mit allen an diese Umgebung gebundenen
Symbolen und überwacht ihre Lebenszeit (siehe \ref{sec:shared_ptr}).

\projectname{} wurde entworfen um eine einfache Schnittstelle zwischen einem C++-Programm
und lisp-programmen zu schaffen. Aus diesem Grund hält \projectname{} durchgehend
eine sogenannte globale Umgebung. Diese Umgebung ist eine Instanz der hier genannten 
!environment! Klasse. Das Client-C++-Programm sollte alle geladenen lisp-programme
über diese globale Umgebung ausgeführen. Damit wird eine bequeme Weitergabe von Objekten
und Funktionen an die Lisp-Seite ermöglicht in dem diese von jeder Stelle aus an Symbole
aus der globalen Umgebung gebunden werden können.

Im aktuellen Entwicklungsstand werden auch die durch den Kern implementierten
Formen auf diesem Weg ausgeliefert (siehe \ref{sec:cxx_interface}).
